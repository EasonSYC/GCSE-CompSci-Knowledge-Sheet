\documentclass[8pt]{article}
\usepackage{allan-eason}

\usepackage{siunitx}

\usetikzlibrary{positioning}
\usetikzlibrary{svg.path}

\graphicspath{ {./images/} }


\newcommand{\Date}{\today}
\newcommand{\Name}{GCSE CS Revision Sheet}
\newcommand{\Subname}{Eason's Toolbox}
\newcommand{\Title}{\Name\\ \Large{\Subname}}

\newcommand{\Author}{Yicheng Shao}

\theoremstyle{remark}

\author{\Author}
\title{\Title}
\date{Version 1. \Date}

\lhead{\Name}

\newtheorem{knowledge}{Knowledge}[subsection]

\begin{document}

	\maketitle

	\tableofcontents

    \section*{What is this and why this?}
    
        GCSE Computer Science is a knowledge-intense exam. Unlike what most people think, Computer Science is a subject that requires a lot of writing. Therefore, I made this document based on the CIE IGCSE Computer Science (9-1) Syllabus from 2023 onwards. I hope this could help with you IGCSE studies!

        This is more of an extension of the syllabus and the structure is exactly the same. However, it provides some sample answers for those questions in the syllabus and is a good way to refer to your self assessment based on the syllabus.

        I am also an IGCSE student so errors are inevitable in this document. Feel free to email \href{eason.syc@icloud.com}{eason.syc@icloud.com} to point out any mistakes or submit an issue on the GitHub page!
        
    \section{Data Representation}
        \subsection{Number Systems}

        \begin{knowledge}
            \textbf{Analogue data} is continuous. \textbf{Digital data} is discrete.
        \end{knowledge}

        \begin{knowledge}
            \textbf{Denary} is a base-10 number system. \textbf{Binary} is a base-2 number system. \textbf{Hexadecimal} is a base-16 number system.
        \end{knowledge}

        \begin{knowledge}
            Binary is required as computers process data using logic gates and registers.
        \end{knowledge}

        \begin{knowledge}
            Digits in decimal, hexadecimal, binary can be converted as
            \begin{center}
                \begin{tabular}{r|c|c|c|c}
                    \textbf{Denary} & 0 & 1 & 2 & 3\\\hline
                    \textbf{Hexadecimal} & 0 & 1 & 2 & 3\\\hline
                    \textbf{Binary} & 0000 & 0001 & 0010 & 0011\\\hline\hline
                    \textbf{Denary} & 4 & 5 & 6 & 7\\\hline
                    \textbf{Hexadecimal} & 4 & 5 & 6 & 7\\\hline
                    \textbf{Binary} & 0100 & 0101 & 0110 & 0111\\\hline\hline
                    \textbf{Denary} & 8 & 9 & 10 & 11\\\hline
                    \textbf{Hexadecimal} & 8 & 9 & A & B\\\hline
                    \textbf{Binary} & 1000 & 1001 & 1010 & 1011\\\hline\hline
                    \textbf{Denary} & 12 & 13 & 14 & 15\\\hline
                    \textbf{Hexadecimal} & C & D & E & F\\\hline
                    \textbf{Binary} & 1100 & 1101 & 1110 & 1111\\
                \end{tabular}
            \end{center}
        \end{knowledge}
        
        \begin{knowledge}
            To convert binary into hexadecimal, we map four consecutive digits (divide from the right) to one digit in hexadecimal. Vice versa.
        \end{knowledge}

        \begin{knowledge}
            To convert binary or hexadecimal into denary, we write the digit it represents on top (e.g. 2, 4, 8 or 16, 256, 4096), and times it with the number below and sum them together. Write from right to left.
        \end{knowledge}

        \begin{knowledge}
            To convert denary into binary or hexadeximal, we write down the result of integer division of the number and the base, and use the remainder to continue. When we get a remainder that is less than the base, we write the result backwards.
        \end{knowledge}

        \begin{knowledge}
            Programmers use hexadecimals as it is easier for human to read. Examples include MAC code.
        \end{knowledge}

        \begin{knowledge}
            Binary addition works similarly as denary addition.
        \end{knowledge}

        \begin{knowledge}
            \textbf{Overflow error} stands for when data is too big when stored in certain amounts of digits. It could happen in binary addition or binary shifts.
        \end{knowledge}

        \begin{knowledge}
            \textbf{Left shift} is defined by shifting all the digits to the left. \textbf{Right shift} is defined similarly.

            We fill in the empty slots with 0 and simply delete the slots that went out.
        \end{knowledge}

        \begin{knowledge}
            Left shift is timing the original number by the base. Right shift is (integer) dividing it.
        \end{knowledge}

        \begin{knowledge}
            \textbf{Negative binary is stored as follows.}
            \begin{itemize}
                \item Find the binary of the original number (i.e. non-negative).
                \item Invert all the 0s and 1s.
                \item Add this by 1 (treat it as a positive number).
            \end{itemize}
            This is called \textbf{two's complement}.
        \end{knowledge}

        \subsection{Text, Sound and Image}

        \begin{knowledge}
            We store text using \textbf{character sets}. ASCII code and Unicode are two examples. Unicode are more universal but is bigger storing single character.
        \end{knowledge}

        \begin{knowledge}
            An \textbf{image} is a series of \textbf{pixels} that are converted to binary. The \textbf{resolution} is the number of pixels in an image. The \textbf{colour depth} is the number of bits used to store each colour. There is \textbf{metadata} at the beginning of a file to state those information. The file size and quality of the image increases as the resolution and colour depth increase.
        \end{knowledge}

        \begin{knowledge}
            A \textbf{sound sampling} is done to convert analogue sound data into digital data. The \textbf{sample rate} is the number of samples taken in a second. The \textbf{sample resolution} is the number of bits per sample. The accuracy of the recording and the file size increases as the sample rate and resolution increase.
        \end{knowledge}

        \subsection{Data Storage and Compression}

        \begin{knowledge}
            Following are required data representation units:
            \begin{itemize}
                \item A \textbf{bit} is a binary digit.
                \item A \textbf{nibble} is four bits.
                \item A \textbf{byte} is eight bits.
                \item A \textbf{kibibyte} (KiB) is 1024 bytes.
                \item A \textbf{mebibyte} {MiB} is 1024 kibibytes.
                \item A \textbf{gibibyte} {GiB} is 1024 mebibytes.
                \item A \textbf{tebibyte} {TiB} is 1024 gibibytes.
                \item A \textbf{pebibyte} {PiB} is 1024 tebibytes.
                \item A \textbf{exbibyte} {EiB} is 1024 pebibytes.
            \end{itemize}
        \end{knowledge}

        \begin{knowledge}
            Way to calculate size of image:
            \[
                \text{size of image} = \text{width of image} \times \text{height of image} \times \text{colour depth of image} \times \text{number of images in file}.
            \]
        \end{knowledge}

        \begin{knowledge}
            Way to calculate size of sound track:
            \[
                \text{size of soundtrack} = \text{sample rate} \times \text{sample resolution} \times \text{length of soundtrack}.
            \]
        \end{knowledge}

        \begin{knowledge}
            \textbf{Data compression} is the method used to reduce the size of a file. It is necessary as
            \begin{itemize}
                \item it will require smaller storage;
                \item it will take less time to transmit;
                \item it will be quicker to upload and download;
                \item it will require smaller bandwidth.
            \end{itemize}
        \end{knowledge}

        \begin{knowledge}
            \textbf{Lossy} compression can not be reversed, it permantly removes unnecessary and redundant data in file. Examples to compress include reducing resolution, reducing colour depth, and reducing sample rate. Examples of files include .jpg, .mp3.
        \end{knowledge}

        \begin{knowledge}
            \textbf{Lossless} compression reduces size without loss of information. Examples to compress include run length encoding (RLE) which groups together repeating data. Examples of files include .midi.
        \end{knowledge}

    \section{Data Transmission}

        \subsection{Types and Methods of Data Transmission}    

        \begin{knowledge}
            Data is transmitted in small units called \textbf{packets}. They are divided into \textbf{packet header}, \textbf{payload} and \textbf{trailer}.
        \end{knowledge}
        
        \begin{knowledge}
            Packet header includes the destination address, packet number and originators address. The addresses are often IP addresses.
        \end{knowledge}

        \begin{knowledge}
            The payload is the actual data you are sending.
        \end{knowledge}

        \begin{knowledge}
            The trailer (a.k.a. footer) indicates end of packet and the error detection systems used.
        \end{knowledge}

        \begin{knowledge}
            \textbf{Packet switching} is the process of transmitting packets over a \textbf{network} with \textbf{routers}. The following is the process:
            \begin{itemize}
                \item Data is broken down into packets;
                \item Each packet could take a different route;
                \item A router controls the route a packet takes;
                \item Packets may arrive out of order;
                \item Once the last packet has arrived, packets are reordered.
            \end{itemize}
        \end{knowledge}

        \begin{knowledge}
            Data transmission can be divided into two types, \textbf{serial} and \textbf{parallel}.
        \end{knowledge}

        \begin{knowledge}
            Serial transmission is transmitted one bit at a time along one wire. Parallel transmission is transmitted multiple bits at a time along multiple wires.

            \textbf{Advantages of Serial Transmission:} (which relates to a disadvantage of parallel transmission)
            \begin{itemize}
                \item The data is transmitted in sequence, so there is less chance of data being skewed.
                \item The data is transmitted along a single wire hence less chance of \textbf{interference}, with less chance of error.
                \item Only one wire is necessary so it is cheaper.
            \end{itemize}

            \textbf{Advantages of Parallel Transmission:} (which relates to a disadvantage of series transmission) Data transmission is faster since multiple bits are sent at a time.

            \textbf{Disadvantage for serial:} a start bit and an end bit is necessary.

            \textbf{Advantage for parallel:} No requirement to convert data across network.
        \end{knowledge}

        \begin{knowledge}
            Data transmission can be divided into three types, \textbf{simplex}, \textbf{half-duplex} and \textbf{duplex}.
        \end{knowledge}

        \begin{knowledge}
            \textbf{Simplex} transmission is when data transmits in only one direction. \textbf{Half-duplex} is when data can be transmitted bi-directionally but only one direction at a time. \textbf{Full-duplex} is when data can be transmitted bi-directionally simultaniously.
        \end{knowledge}

        \begin{knowledge}
            The \textbf{USB}, or universal series bus, is a standard (protocol) for data transmission.
        \end{knowledge}

        \begin{knowledge}
            \textbf{Advantages for USB}:
            \begin{itemize}
                \item It is a very simple interface, and a very low rate of error (it is probably idiot-proof).
                \item The speed is relatively high.
                \item It is a very universal standard and widely used.
                \item It is automatic in terms of detection while inserted, including downloading of drivers.
                \item It can be used as a power source.
            \end{itemize}

            \textbf{Disadvantages for USB}:
            \begin{itemize}
                \item USB is length-limited.
                \item USB is not as fast as certain interfaces such as \textbf{ethernet} (and thunderbolt, PCIe, SATA).
            \end{itemize}
        \end{knowledge}

        \subsection{Methods of Error Detection}

        \begin{knowledge}
            \textbf{Error detection} is necessary since during transmission data can be interfered, such as data loss, data gain and data change.
        \end{knowledge}
        
        \begin{knowledge}
            \textbf{Parity check} can be odd or even (which is determined previously). The odd or even stands for the number of 1s in the data. \textbf{Parity block check} and \textbf{parity byte} is also used (vertical and horizontal parity checks) to detect and potentially recover data. 
        \end{knowledge}

        \begin{knowledge}
            \textbf{Checksum} is using a calculated value to check for errors. This value is calculated from the data before transmission and compared with the result of the value after transmissino. Examples include modulus 11 (md5, sha1, shaxxx).
        \end{knowledge}

        \begin{knowledge}
            \textbf{Echo check} is when data is transmitted back to the sender to check.
        \end{knowledge}

        \begin{knowledge}
            A \textbf{check digit} is like a checksum and parity combined. The checksum data is included in the data itself. Examples include ISBN and bar codes. 
        \end{knowledge}

        \begin{knowledge}
            \textbf{ARQ} is automatic repeat query when data can be repeatedly transmitted when error occurs. It involves \textbf{acknowledgement} and \textbf{timeout}.

            \textbf{A positive acknowledgement involves:}
            \begin{itemize}
                \item The sending device transmits the first data packet;
                \item The receiving device receives the data and checks it ofr errors;
                \item If it does not have error: it sends a positive acknowledgement and the sender continues sending the next packet.
                \item If the sender does not receive the positive acknowledgement within the set timeframe, this is a timeout, and it will continue sending the same packet until a positive acknowledgement occurs or a limit is reached.
            \end{itemize}

            \textbf{A negative ackowledgement involves:}
            \begin{itemize}
                \item The sending device transmits the first data packet;
                \item The receiving device receives the data and checks it ofr errors;
                \item If it does not have error: no further action. The sender begins to send the next packet after the set time period.
                \item If it does have an error: A negative acknowledgement is sent. The sender will resend the data.
            \end{itemize}
        \end{knowledge}

        \subsection{Encryption}
            \begin{knowledge}
                \textbf{Encryption} is a method of data protection while data is transmitted since hackers may try and intercept the data while transmitting.
            \end{knowledge}

            \begin{knowledge}
                The \textbf{plain text} is the original text, the \textbf{encryption key} is the method used to encrypt it into the \textbf{cipher text} which seems meaningless.
            \end{knowledge}

            \begin{knowledge}
                \textbf{Symmetric} encryption is when the encryption and decryption uses the same key.
                \begin{itemize}
                    \item The plain text is encrypted into cipher text using an encryption key;
                    \item The cipher text and the encryption key are sent separately to the receiving device;
                    \item The same key is then used to decrypt the cipher text back into its plain text form.
                \end{itemize}

                \textbf{Asymmetric} encryption is when public key (encryption) and private key (decryption) are used separately.
                \begin{itemize}
                    \item The plain text is encrypted into cipher text using a public key;
                    \item The cipher text is transmitted to the receiving device;
                    \item The ciphter text is decrypted with a private key.
                \end{itemize}

                The reverse (private for encrypting and public for decrypting) can be worked for digital signatures, e.g. SSL.
            \end{knowledge}

    \section{Hardware}
        
        \subsection{Computer Architecture}
            \begin{knowledge}
                \textbf{CPU}, central processing unit processes instructions and data that are input into the computer so that the result can be output. \textbf{Microprocessor} is a type of integrated circuit on a single chip.
            \end{knowledge}

            \begin{knowledge}
                Components in a von-Neumann CPU include:
                \begin{center}
                    \begin{tabular}{c|p{0.5\linewidth}}
                        Control Unit (CU) & A component that sends signals to control the interactions of all other components during the fetch-decode-execute cycle. \\\hline\hline
                        Arithmetic Logic Unit (ALU) & A component that performs all calculations and logical operations required during the fetch-decode-execute cycle.\\\hline\hline
                        Address Bus & A bus that is used to transmit addresses within the CPU and to, and from RAM.\\\hline
                        Data Bus & A bus that is used to transmit data or instructions within the CPU, and to and from RAM.\\\hline
                        Control Bus & A bus that is used to transmit control signals that are sent by the control unit.\\\hline\hline
                        Memory Address Register (MAR) & A register that stores the address of where data or an instruction is located in RAM.\\\hline
                        Memory Data Register (MDR) & A register that stores data or an instruction when it is fetched from RAM.\\\hline
                        Program Counter (PC) & A register that holds the address of the next instruction to be processed.\\\hline
                        Current Instruction Register (CIR) & A register that holds the instruction that is currently being processed.\\\hline
                        Accumulator (ACC) & A register that is built into the ALU that is used to store the interim results of calculations.
                    \end{tabular}
                \end{center}
            \end{knowledge}

            \begin{knowledge}
                \textbf{The fetch-decode-execute} cycle is the process in which CPU processes instructions (which usually adjusts data).

                \begin{itemize}
                    \item The data within the PC is sent to the MAR.
                    \item PC adds itself by one instruction.
                    \item The MAR data is sent to the RAM via the Address Bus.
                    \item The data within that address is sent to the MDR via the Data Bus.
                    \item This data (the instruction) is sent to the CIR.
                    \item The CPU decodes the instruction using an instruction set. (A set of all commands)
                    \item The CPU fetches all data similarly but process them with ALU (ACC is also involved for loops).
                \end{itemize}

                The control bus is involved in the fetch stage while data is transmitted via the two other buses.
            \end{knowledge}

            \begin{knowledge}
                The \textbf{core} (which is a unit of the FDE cycle execution), the \textbf{clock speed} (which is a speed where instructions are processed), and the \textbf{cache} (which is a type of RAM-like volatile storage but smaller and quicker) will affect the performance of a CPU.
            \end{knowledge}

            \begin{knowledge}
                An \textbf{instruction set} is a list of all the commands that can be processed by a CPU and the commands are machine code
            \end{knowledge}

            \begin{knowledge}
                An \textbf{embedded system} is used to perform a dedicated function, e.g. domestic appliances, cars, security systems, lighting systems or vending machines. This is different to a \textbf{general purpose computer} that is used to perform many different functions, e.g. a personal computer (PC) or a laptop
            \end{knowledge}

        \subsection{Input and output devices}

            \begin{knowledge}
                An \textbf{input device} is a device that allows data to be entered into a computer system.
            \end{knowledge}

            \begin{knowledge}
                Input devices include:
                \begin{center}
                    \begin{tabular}{p{0.2\linewidth}|p{0.3\linewidth}|p{0.3\linewidth}}
                        Barcode Scanner & It scans a barcode so that the data stored in the barcode can be obtained. & It is used in a supermarket to get the price of a product and as part of a tock control system.\\\hline
                        Digital Camera & It captures light through a lens and converts it into binary. & It is built into a mobile phone to allow the user to photograph items or people.\\\hline
                        Keyboard & It allows the user to press keys that have a designated ASCII/Unicode value that is converted to binary. & It is one of the main methods of input that allows a user to type data into a personal computer.\\\hline
                        Microphone & It captures soundwaves and converts them to binary. & It is built into a mobile phone to capture the user's voice so that it can be heard by the other users. \\\hline
                        Optical Mouse & It captures the light that is bounced back from a laser that is shone from the mouse to the surface underneath, to track the mouse's movements. & It is one of the main methods of input that allows a user to select icons and menu options whilst using a personal computer.\\\hline
                        QR Code Scanner & It uses a sensor or a camera to capture light reflected from a QR code and converts it to binary. & it can be an application that is downloaded onto a mobile phone and used to scan QR codes that store information, e.g. a website link.\\\hline
                        Scannder (2D and 3D) & They use sensors to capture light that is reflected from a 2D or 3D object and convert it to binary. & It can be used to scan 3D objects to create a digital copy of them.\\\hline
                        Touch Screen (Resistive, Capacitive and Infra-Red) & They use pressure, conductivity or light to register the touch of a user on a screen. The coordinates of the touch can be calculated. & It is built into a ticket machine to allow a user to select which ticket they would like to buy.
                    \end{tabular}
                \end{center}
            \end{knowledge}

            \begin{knowledge}
                An \textbf{output device} is a device that allows the result of the data processing to be seen or heard.
            \end{knowledge}

            \begin{knowledge}
                Output devices include:
                \begin{center}
                    \begin{tabular}{p{0.2\linewidth}|p{0.3\linewidth}|p{0.3\linewidth}}
                        Actuator & It is a component that outputs an action, often a type of movement, that causes another device to operate. & It can be used in an automated system to move or turn on/off another device, e.g. a light.\\\hline
                        Digital Light Processing (DLP) Projector & It is a device that uses light reflected from millions of little mirrors to output an image. & It can be used in a classroom to project an image onto an interactive whiteboard.\\\hline
                        Inkjet Printer & This is a device that squirts liquid ink from nozzles to output a document or image. & Ut can be used in a house to print photographs.\\\hline
                        Laser Printer & This is a device that uses a rotating drum and powdered toner to output a document. & It can be used in an office to print letters.\\\hline
                        Light Emitting Diode (LED) Screen & This is a screen that uses LEDs as a backlight to output an image. & the screen can be built into a mobile phone.\\\hline
                        Liquid Crystal Display (LCD) Projector & This is a device that shines light through crystals and then through a lens to project an image onto a blank wall or screen. & This can be used to project an image in a home cinema system.\\\hline
                        Liquid Crystal Display (LCD) Screen & This is a screen that shines light through crystals to output an image. & This can be built into a television screen.\\\hline
                        Speaker & This is a device that outputs sound. & This can be built into a mobile phone so one user can hear another user's voice.\\\hline
                        3D printer & This is a device that builds layers of material to output a 3D object. & This can be used in medicine to create prosthetic limbs.
                    \end{tabular}
                \end{center}
            \end{knowledge}

            \begin{knowledge}
                A \textbf{sensor} is also an input device, and it is used in an \textbf{automated system}.
            \end{knowledge}

            \begin{knowledge}
                Sensors include:
                \begin{center}
                    \begin{tabular}{p{0.2\linewidth}|p{0.3\linewidth}|p{0.3\linewidth}}
                        Acoustic & This type of sensors measures the level of sound in an environment. & These sensors are used in many applications that involve sound. An acoustic sensor can be used in a security system. It can be placed near a window and constantly measure the level of sound. If it captures a reading that shows a sudden increase in sound, this could mean that the window has been broken and building may be at risk.\\\hline
                        Accelerometer & This type of sensor measures acceleration forces. These may be static forces, such as the continual force of gravity. They can also be dynamic forces, such as those created by movement and vibrations. & These sensors are used in a wide variety of devices. Mobile phones use an accelerometer to know which way up it is faced, to automatically turn the screen on and off. They can be used to monitor for earthquakes, as they can capture the initial vibrations created. They can also be used in cars to sense when a crash has occured, so that airbags can be inflated.\\\hline
                        Flow & This type of sensor measures the amount of liquid, gas or steam that is flowing through or around a certain environment. & These sensors are often used in factories and sites such as nuclear power plants. They make sure that the liquid, gas or steam flows at a constant temperature through an environment, such as a pipe. This makes sure that the pipes don't rupture and break due to too much flowing through them. \\\hline
                        Gas & This type of sensor measures the presence and concentration of a gas within the immediate atmosphere. & These sensors can be used in people's homes. They can be set to measure a certain gas, such as carbon monoxide. They constantly capture the data in the immediate atmosphere to see if too much carbon monoxide is present, which could endanger the health of anyone living in the home.\\\hline
                    \end{tabular}
                    \begin{tabular}{p{0.2\linewidth}|p{0.3\linewidth}|p{0.3\linewidth}}
                        Humidity & This type of sensor measures the level of moisture in the immediate atmosphere. & These sensors can be used in farming and agriculture to make sure that the air in areas such as greenhouses has the correct level of moisture to provide the best growing conditions for the fruits and vegetables. They can also be used in places such as art galleries, to make sure that the humidity level is constant. Too much or too little can ruin paintings.\\\hline
                        Infra-red & This type of sensor measures infra-red radiation. This type of radiation can be emitted in different amounts by both objects and people. & These sensors can also be used in security systems.
                        This can be done in two different ways. The device containing the sensor can emit infra-red radiation and when this bounces back to the device, the readings can show from the distance it has travelled whether an intruder is present. It can also operate by capturing the infra-red radiation emitted by the intruder.\\\hline
                        Level & This type of sensor measures whether a substance, such as a liquid, is at a certain level or amount. & These sensors can be used in a car to make sure that essential liquids, such as oil and fuel, do not get too low.\\\hline
                        Light & This type of sensor measures the ambient light in a certain environment. It can also measure the presence of a direct light, such as a laser beam. & These sensors can be used in automatic lighting systems. Streetlights can be fitted with a light sensor that will allow the light to turn on in the evening, when it becomes darker.\\\hline
                        Magnetic field & This type of sensor measures the presence of magnetic field that may be emitted by an obiect. & These sensors can be used to count how many cars pass through a certain area, for example into a car park. The car will disrupt the Earth's naturally magnetic field as it passes over the sensor and the data can be captured by the sensor.\\\hline
                    \end{tabular}
                    \begin{tabular}{p{0.2\linewidth}|p{0.3\linewidth}|p{0.3\linewidth}}
                        Moisture & This type of sensor measures the amount of water that is present in a substance, such as soil. & These sensors can also be used by farming and agriculture, to make sure that the fruits and vegetables have the best level of water in the soil to help them grow.\\\hline
                        pH & This type of sensor measures the pH level of a substance. & These sensors can be used by environmental agencies to make sure that local lakes and river are not being polluted.\\\hline
                        Pressure & This type of sensor measures the force of pressure that is applied to the sensor or device.
                        This could be the pressure created by a solid object, or it could be created by liquid or gas. & These sensors could also be used in a security system.
                        They can be placed at the base or sides of an opening, such as a window or a door. When that window or door is opened, the pressure will decrease, and the system will recognise that an intruder has entered.\\\hline
                        Proximity & This type of sensor measures how close an object is in comparison to the sensor. It does this by emitting electromagnetic radiation or an electromagnetic field and measures the radiation as it returns to see if there are any changes. & These types of sensors can be used in robots in manufacturing. They allow the robots to measure how close they are to different objects, when moving around a factory.\\\hline
                        Temperature & This type of sensor measures the temperature of an object or substance by either directly touching it or capturing data from the surrounding environment. & These types of sensors are used in air conditioning systems. They allow the temperature of a room to be kept at a certain level.
                    \end{tabular}
                \end{center}
            \end{knowledge}

        \subsection{Data Storage}
            
            \begin{knowledge}
                Storage can be divided into two types, \textbf{primary storage} and \textbf{secondary storage}. Primary is the one that is directly accessed by the CPU, secondary is the one that cannot be directly accessed by the CPU (usually by I/O device).
            \end{knowledge}

            \begin{knowledge}
                \textbf{RAM}, random access memory, and \textbf{ROM}, read-only memory, are two types of primary storage.

                \begin{center}
                    \begin{tabular}{p{0.4\linewidth}|p{0.4\linewidth}}
                        RAM is volatile (temporary). When powered off, the contents are lost. & ROM is non-volatile (permanant). When powered off, the contents remains.\\\hline
                        RAM stores data and programs currently in use. & ROM stores the bootstrap and the BIOS which boots the computer.\\\hline
                        RAM contents are constantly being changed. & ROM data are fixed. (It is possile to change, but it is difficult and we usually do not do it.)\\\hline
                        RAM can be increased by adding components. & ROM are fixed on the motherboard (usually).
                    \end{tabular}
                \end{center}
            \end{knowledge}

            \begin{knowledge}
                Secondary storage are not directly accessed by the CPU (and buses hence slower), but is necessary for larger storage capacity. There are three main types, \textbf{magnetic}, \textbf{optical} and \textbf{solid-state storage}.
            \end{knowledge}

            \begin{knowledge}
                Magnetic storage uses platters which are divided into tracks and sectors. Data is read and written using electromagnets.
                
                Optical storage uses lasers to create and read pits and lands.
                
                Solid-state (flash memory) uses NAND or NOR technology. Transistors are used as control gates and floating gates.
            \end{knowledge}

            \begin{knowledge}
                \textbf{Virtual memory} is the type of memory created for temporary use and is an extension to RAM. \textbf{Pages} of data are exchanged between secondary storage and the RAM. When the CPU needs them they will be transferred to the RAM, and when not it will be swapped back to the secondary storage.
            \end{knowledge}

            \begin{knowledge}
                \textbf{Cloud storage} can be accessed remotely in comparison to storing data locally. Physical servers and storage are needed to store data in cloud storage.
            \end{knowledge}

            \begin{knowledge}
                Cloud or local:
                \begin{center}
                    \begin{tabular}{p{0.4\linewidth}|p{0.4\linewidth}}
                        You do not need to pay for cloud storage hardware. & You need to pay for local storage hardware.\\\hline
                        Yo are not responsible for the security, but you should make sure the third-party service provider is safe. & You need to make sure the storage is regularly checked and updated for security issues.\\\hline
                        You can access data and resources using different devices from anywhere as long as internet is avaliable. However, if connection fails, you cannot access. If connection is slow, access is slow. & You do not need internet connection.\\\hline
                        You can increase/decrease storage capacity as you need. & You might have redundant hardware, more than you need.
                    \end{tabular}
                \end{center}
            \end{knowledge}

        \subsection{Network hardware}
            \begin{knowledge}
                \textbf{NIC}, or network interface card, is required to connect to the internet. \textbf{MAC} address, or media access control address, is paired with a NIC to identify a device on a network. It is assigned by the manufacturer and never changed. It includes a 14-digit hexadecimal with 7 parts using colons (:) to separate them. It iscreated using the manufacturer code and the serial code. 
            \end{knowledge}

            \begin{knowledge}
                A \textbf{router} sends data to a specific destination on a network, assigns IP addresses, and can connect a local network to the
                internet.
            \end{knowledge}

            \begin{knowledge}
                An \textbf{IP address} is allocated by the network and they can be static or dynamic. \textbf{Dynamic} IP addresses are most common and they are assigned every time a device connects to a network. \textbf{Static} IP address is usually provided by the Internet Service Provider and will be the same every time you connect. They are unique and can also be used to identify a device on a network. \textbf{IPv4} consists of 32-bits IP address, with 4 numbers (decimal) and full stops . separating them. \textbf{IPv6} consists of 128-bits (32-digit hexadecimal) with colons separating them and is capable of creating more IP addresses.
            \end{knowledge}
    
    \section{Software}

        \subsection{Types of Software and Interrupts}
            \begin{knowledge}
                \textbf{Softwares} can be divided into two categories, \textbf{System Software} and \textbf{Application Software}.
            \end{knowledge}

            \begin{knowledge}
                \textbf{System Software} provides the services that the computer requires, including operating system and utility software. \textbf{Application Software}provides the services that the user requires.
            \end{knowledge}

            \begin{knowledge}
                Functions of an \textbf{Operating System} include:
                \begin{itemize}
                    \item managing files,
                    \item handling interrupts,
                    \item providing an interface (e.g. graphical GUI, command line, natural language),
                    \item managing peripherals (I/O devices) and drivers,
                    \item managing memory,
                    \item managing multitasking (by interrupts),
                    \item providing a platform for running applications,
                    \item providing system security, and
                    \item managing user accounts.
                \end{itemize}
            \end{knowledge}

            \begin{knowledge}
                Examples of \textbf{Application Softwares} include:
                \begin{itemize}
                    \item word processor,
                    \item spreadsheet,
                    \item database, and
                    \item web browser.
                \end{itemize}
            \end{knowledge}

            \begin{knowledge}
                \textbf{Application software} runs on \textbf{operating system}, which runs on \textbf{firmware}, which is loaded by the \textbf{bootstrap}, which directly runs on the hardware. 
            \end{knowledge}

            \begin{knowledge}
                An \textbf{interrupt} is a signal to tell the processor to tell it that something needs attention. Two types are \textbf{Software Interrupt} and \textbf{Hardware Interrupt}.
            \end{knowledge}

            \begin{knowledge}
                Examples of software interrupts include:
                \begin{itemize}
                    \item Division by 0;
                    \item Two processes attempting to access the same memory location;
                    \item Request for input;
                    \item Output required;
                    \item Data required from memory.
                \end{itemize}

                Examples of hardware interrupts include:
                \begin{itemize}
                    \item Data input (e.g. keyboard input/mouse click);
                    \item Error from hardware (e.g. printer out of paper);
                    \item Hardware failure;
                    \item Hard drive signal that it has read data;
                    \item New hardware device connected.
                \end{itemize}
            \end{knowledge}

            \begin{knowledge}
                Interrupts are handled by an \textbf{Interrupt Handler (IH)} with an \textbf{Interrupt Service Routine (ISR)}. The process is as follows:
                \begin{itemize}
                    \item When the CPU finishes an FDE cycle it checks the interrupt queue.
                    \item It checks whether there is an interrupt with higher priority than the current task.
                    \item If yes,
                    \begin{itemize}
                        \item It stores the current process and fetches the interrupt.
                        \item It checks the source of the interrupt.
                        \item It calls (executes) the relative ISR which handles the interrupt.
                        \item The stored process is returned to the memory, or a higher-priority interrupt is fetched.
                    \end{itemize}
                    \item If not, it runs another FDE cycle.
                \end{itemize}
            \end{knowledge}

        \subsection{Types of Programming Language, Translators and Integrated Development Environments (IDEs)}
            \begin{knowledge}
                Programming languages include two types, \textbf{high-level} and \textbf{low-level}.
            \end{knowledge}
            
            \begin{knowledge}
                A high-level programming language uses human-style words for instructions. High-level programming languages are \textbf{portable} -- you can write it on one device and run it on another.
            \end{knowledge}

            \begin{knowledge}
                Low-level programming languages can be further divided into \textbf{machine code} and \textbf{assembly language}. Machine code, as its name suggests, is machine-specific i.e. it is \textbf{non-portable}. Assembly uses \textbf{mnemonics} to represent code which is in the middle.
            \end{knowledge}

            \begin{knowledge}
                High-level language v.s. low-level language:
                \begin{center}
                    \begin{tabular}{p{0.4\linewidth}|p{0.4\linewidth}}
                        High-level languages & Low-level languages\\\hline\hline
                        Easier for users to understand, read, write and ament. & More difficult to do so.\\\hline
                        Easier to debug. & More difficult to do so.\\\hline
                        Machine independent (portable). & Machine dependent (non-portable).\\\hline
                        It must be converted to a low-level language to run. & Machine code does not require converting, assembly needs to be assembled but is significantly faster than the execution of a high-level language.\\\hline
                        One statement can represent many low-level instructions. & Multiple statements are required to represent just one high-level statement.\\\hline
                        Cannot directly manipulate the hardware. & Can directly manipulate hardware, e.g. writing to specific memory locations, which makes it more time and space efficient.
                    \end{tabular}
                \end{center}
            \end{knowledge}

            \begin{knowledge}
                A \textbf{translator} is required to translate programs to machine code to execute.
            \end{knowledge}

            \begin{knowledge}
                In a high-level language, you can use \textbf{interpreter} or a \textbf{compiler} to translate the instructions.

                An \textbf{interpreter} translates and execute the code line-by-line. It stops where there is an error. It is more useful for program writing, but not for whole-system testing, since it needs to interpret everything every time code is executed.

                A \textbf{compiler} translates the whole file all at once and produces an \textbf{executable file} which can be directly executed. It produces an error report of the whole program. It is suitable for whole-program testing.
            \end{knowledge}

            \begin{knowledge}
                Interpreters v.s. compilers:
                \begin{center}
                    \begin{tabular}{p{0.4\linewidth}|p{0.4\linewidth}}
                        Interpreter & Compiler\\\hline\hline
                        Translates one line of code into machine code and then executes it. & Translates all lines of code into machine code, before executing the program. \\\hline
                        Reports a syntax error as soon as it is picked up and stops the program until it is corrected. & Reports all syntax error at the same time, the program is not run until all errors are corrected. \\\hline
                        Useful when writing a program. & Useful when a program has been finished and is ready for testing or distribution. \\\hline
                        Code needs to be re-translated each time the program is run. & Code does not need re-translating.\\\hline
                        Does not produce an executable file. & Produces an executable file.\\\hline
                        Source code is required to run. & Source code is not required.\\\hline
                        Interpreter software is required to run. & No other software is required.\\\hline
                        Partially testing is avaliable. & Whole section of code must be completed to test.
                    \end{tabular}
                \end{center}
            \end{knowledge}

            \begin{knowledge}
                An \textbf{Integrated Development Environment (IDE)} is an application software for you to write and test the code, including most/all software necessary. It includes an \textbf{editor}, the \textbf{translator} and also the \textbf{run-time environment} where the interface is shown while running.
            \end{knowledge}

            \begin{knowledge}
                IDE functions include:
                \begin{itemize}
                    \item code editors;
                    \item run-time environment;
                    \item translators;
                    \item error diagnostics;
                    \item auto-completion;
                    \item auto-correction; and
                    \item prettyprint.
                \end{itemize}
            \end{knowledge}

    \section{The internet and its uses}
        
        \subsection{The internet and the world wide web}

            \begin{knowledge}
                The \textbf{internet} is the infrastructure, especially the cable. It is just a type of WAN which is very special that covers the whole world. The \textbf{world wide web} is the collection of \textbf{websites} and \textbf{web pages} accessed using the internet
            \end{knowledge}

            \begin{knowledge}
                A \textbf{URL} (uniform resource locator) is a text-based address for a web page. It contains the protocol, the domain name and the web page/file name.
            \end{knowledge}

            \begin{knowledge}
                \textbf{HTTP} (hypertext transmission protocol) is a protocol used to transmit requests and the results of the requests between the web browser and the web server.
                
                \textbf{HTTPS} (hypertext transfer protocol secure) adds a layer of security by encrypting data using \textbf{digital certificates} where owners will need to apply from a \textbf{certificate authority}.
            \end{knowledge}

            \begin{knowledge}
                The \textbf{domain name server (DNS)} is a special kind of server that stores the domain names with their equivilant IP address as a form of database.
            \end{knowledge}

            \begin{knowledge}
                The following is the process involved in retrieving web pages (for a standard HTTP protocol):
                \begin{itemize}
                    \item The user opens the web browser and enters the URL into the \textbf{web browser}.
                    \item The web browser sends the URL to the \textbf{DNS}.
                    \item The DNS server returns the IP back to the web browser.
                    \item The web browser receives the IP address and uses the HTTP protocol to request the website to the web server.
                    \item The webserver returns \textbf{Hypertext Markup Language (HTML), Cascading Style Sheets (CSS) and Activescript (JavaScript)} back to the browser using \textbf{HTTP}.
                    \item The web browser then renders the website using those data.
                \end{itemize}

                HTTPS works similarly.
                \begin{itemize}
                    \item Before the web browser send request to ask for the web pages, it asks for a digital certificate.
                    \item The web server sends a certificate to the web browser.
                    \item The web browser checks whether the certificate is authentic.
                    \item If yes, the web browser will continue transmission just as before but encrypted.
                    \item If no, the web browser will report that the website is not secure.
                \end{itemize}
            \end{knowledge}

            \begin{knowledge}
                HTTPS works with a layer of security which may be \textbf{SSL (Secure Sockets Layer)} or \textbf{TLS (Transport Layer Security)}. These are the protocols used to encrypt.
            \end{knowledge}

            \begin{knowledge}
                As described before, the web browser \textbf{renders} the data to let you see the webpage. Furthermore, it provides the following functions:
                \begin{itemize}
                    \item storing bookmarks and favourites,
                    \item recording user history,
                    \item allowing use of multiple tabs,
                    \item storing cookies,
                    \item providing navigation tools, and
                    \item providing an address bar.
                \end{itemize}
            \end{knowledge}

            \begin{knowledge}
                Browsers use \textbf{cookies} to store certain data. Data stored includes
                \begin{itemize}
                    \item saving personal details,
                    \item tracking user preferences,
                    \item holding items in an online shopping cart, and
                    \item storing login details.
                \end{itemize}
            \end{knowledge}

            \begin{knowledge}
                Cookies can be divided into two types, \textbf{session cookies} and \textbf{persistent cookies}. The former one refers to cookies that will be deleted when web page closes, while the latter one will be stored until an expiration data.
            \end{knowledge}

        \subsection{Digital currency}
            \begin{knowledge}
                A \textbf{digital currency} is one that only exists electronically and exchanged by computers. Examples of use include \textbf{credit cards}, \textbf{mobile phone}, \textbf{smart watches}.
            \end{knowledge}

            \begin{knowledge}
                One type of digital currency is called \textbf{cryptocurrency}, which is a digital currency that is managed by a delocalised systems (no central authority) -- bitcoin is a typical example of it. It is encrypted to prove that the transaction exists.
            \end{knowledge}

            \begin{knowledge}
                \textbf{Blockchain} is used to keep track of the payments, which is a chain of blocks/records that shows all transactions for a specific currency.
                
                It uses a \textbf{digital ledger}, which is a public record of all payments made with cryptocurrency. The records cannot be altered since the whole ledger is encrypted, but each record has a unique digital signature with time and date to prove that it exists.
            \end{knowledge}

        \subsection{Cyber security}
            \begin{knowledge}
                \textbf{Brute force attack} is attempts to guess a password by trying all possibilities (all combinations). It is an automated process.

                Solutions include
                \begin{itemize}
                    \item Strong passwords,
                    \item Limiting number of login attempts,
                    \item Biometric 'passwords', and
                    \item 2 Factor Authentication (2FA).
                \end{itemize}
            \end{knowledge}

            \begin{knowledge}
                \textbf{Data interception} is a process using pocket sniffer software to intercept data packets as they move through a network.

                Solutions include
                \begin{itemize}
                    \item Use encryption SSL/TLS (HTTPS).
                \end{itemize}
            \end{knowledge}

            \begin{knowledge}
                \textbf{Distributed denial of service (DDoS) attack} is a process of sending too many requests to a server in an attempt to make it crash. Distributed stands for requests being sent by a network of computers (botnet) infected by malware, and such computers are called 'Bot's/zombies.

                Solutions include
                \begin{itemize}
                    \item Setting up a \textbf{proxy server} which acts like a filter/firewall of requests to prevent the real web server from crashing.
                    \item Using \textbf{anti-malware} software to provent your computer becoming a bot.
                \end{itemize}
            \end{knowledge}

            \begin{knowledge}
                \textbf{Hacking} stands for the act of trying to gain unauthorised access to data by exploiting a vulnerability, and the person acting is called a hacker.

                Solutions include
                \begin{itemize}
                    \item Firewall, which manages outgoing and ingoing connections;
                    \item Automatic OS updates;
                    \item Strong passwords; and
                    \item 2FA.
                \end{itemize}
            \end{knowledge}

            \begin{knowledge}
                \textbf{Malware} stands for the malicious software designed to disrupt a computer or its data.

                Examples include
                \begin{center}
                    \begin{tabular}{c|p{0.7\linewidth}}
                        Virus & Downloaded onto your hard drive which replicates itself and corrupts stored data or uses up all avaliable memory, causing it to slow down or crash.\\\hline
                        Worm & Similar to a virus, but looks for vulnerability holes in a network to use to replicate itself, and will clog up the bandwith of a network and slow it down.\\\hline
                        Spyware & Downloaded onto your hard drive and designed to record actions on computer, such as keyloggers which records all key presses. Data is sent to the perpetrator where  it is analysed to identify patterns in data, which could reveal passwords. This could allow access to online accounts for fraud and identity theft.\\\hline
                        Trojan Horse & A computer software that is used to disguise a malware. It is designed to look harmless (e.g. application or game) but contains a virus/worm.\\\hline
                        Adware & A computer program designed to create pop up and banner adverts when online, whichcan be irritating and frustrating. When you click on the links the creaters are given money.\\\hline
                        Ransomware & It is designed to encrypt your data and stop you from gaining access to it, and the creaters will demand a ransom for data to be decrypted. They will threaten to release and leak your stored data if ransom is not paid.
                    \end{tabular}
                \end{center}
                
                Solutions include
                \begin{itemize}
                    \item encrypting data,
                    \item data trackup,
                    \item firewall, and
                    \item anti-malware software.
                \end{itemize}

                This is an example of an effective cybe security systems consisting of multiple layers.
            \end{knowledge}
            
            \begin{knowledge}
                \textbf{Pharming} is a malware that directs you to a fake website when you enter a genuine URL.

                Solutions include
                \begin{itemize}
                    \item anti-malware software, and
                    \item users visually checking websites.
                \end{itemize}
            \end{knowledge}

            \begin{knowledge}
                \textbf{Phishing} is the process of sending an email that encourages you to click on a link to a fake website.

                Solutions include
                \begin{itemize}
                    \item Delete.
                \end{itemize}
            \end{knowledge}

            \begin{knowledge}
                \textbf{Social engineering} is an attempt to manipulate or decieve people to release security information.

                Solutions include
                \begin{itemize}
                    \item access control,
                    \item high privacy level on social media, and
                    \item awareness.
                \end{itemize}
            \end{knowledge}

            \begin{knowledge}
                You are expected to know the following solutions and come up with a solution to a given situation:
                \begin{itemize}
                    \item access levels
                    \item anti-malware including anti-virus and anti-spyware
                    \item authentication (username and password, biometrics, two-step verification)
                    \item automating software updates
                    \item checking the spelling and tone of communications
                    \item checking the URL attached to a link
                    \item firewalls
                    \item privacy settings
                    \item proxy-servers
                    \item secure socket layer (SSL) security protocol.
                \end{itemize}
            \end{knowledge}

            \section{Automated and emerging technologies}

        \subsection{Automated Systems}
            \begin{knowledge}
                An \textbf{automated system} performs actions without the interference with humans and are widely used. It usually involves sensors, microprocessors and actuators (which are introduced previously).
            \end{knowledge}

            \begin{knowledge}
                An automated system works as follows:
                \begin{itemize}
                    \item Sensors retrieve data from the environment and outputs analogue signal to the ADC.
                    \item The ADC (analogue-to-digital) converter converts this signal to digital and outputs to microprocessor.
                    \item The microprocessor reads the data, compare with stored values (or a simple algorithm) and make a decision of whether and how to use the actuator to affect the environment.
                    \item It outputs a signal to the DAC (digital-to-analogue) converter and is converted to analogue signal.
                    \item The actuators receive the analogue signal and causes something to happen, e.g. opening a door, switching on a fan/a heater.
                \end{itemize}

                The above described process is repeated.
            \end{knowledge}

            \begin{knowledge}
                This is an example of a \textbf{closed feedback loop}:
                \begin{itemize}
                    \item The use of an actuator changes the environment;
                    \item the change in environment is measured by sensors; and
                    \item the sensor readings are used to influence future decisions about the use of the actuator.
                \end{itemize}
            \end{knowledge}

            \begin{knowledge}
                Do consider the following while asked for automated systems:
                \begin{itemize}
                    \item Initial cost (the cost for the infrastructure),
                    \item running cost (e.g. cost of electricity and maintenance),
                    \item safety (usually safer since people may be distracted),
                    \item replacing people's jobs (usually both replaced and created),
                    \item continous work all day every day (the result of it), and
                    \item precision (usually higher since no human errors are made).
                \end{itemize}

                The given scenarios will be some that you are familiar with, i.e. limited to
                \begin{itemize}
                    \item industry,
                    \item transport,
                    \item agriculture,
                    \item weather,
                    \item gaming,
                    \item lighting, and
                    \item science.
                \end{itemize}
            \end{knowledge}

        \subsection{Robotics}

            \begin{knowledge}
                \textbf{robotics} is the science of the design, construction and operation of the robots.
                
                A \textbf{robot} is:
                \begin{itemize}
                    \item programmable,
                    \item semi-autonomous (need human supervision only),
                    \item mechanical device that performs an action (often human action),
                    \item using sensors to measure the environment,
                    \item using actuators to move themself/other objects, and
                    \item using microprocessors to control what they do.
                \end{itemize}
            \end{knowledge}

            \begin{knowledge}
                Robots may be used in multiple fields, including (with examples):
                \begin{itemize}
                    \item industry e.g. print spraying,
                    \item transport e.g. driverless trains,
                    \item agriculture e.g. moving lawns,
                    \item medicine e.g. surgery,
                    \item domestic e.g. washing machines and drones,
                    \item entertainment e.g. dance, music and light.
                \end{itemize}
            \end{knowledge}
        \subsection{Artificial Intelligence}

            \begin{knowledge}
                \textbf{Artificial Intelligence (AI)} is the development of programs to simulate intelligent human behaviour, with certain functions such as:
                \begin{itemize}
                    \item image recognition,
                    \item speech recognition,
                    \item natural language,
                    \item computer games, and
                    \item diagnostis systems.
                \end{itemize}
            \end{knowledge}

            \begin{knowledge}
                Four key features of AI are:
                \begin{itemize}
                    \item collection of data,
                    \item a set of programmed rules,
                    \item the ability to reason, and
                    \item the ability to learn and adapt. This may not be true for all AI systems but ML (machine learning) will do so as we will explain in the next point.
                \end{itemize}
            \end{knowledge}

            \begin{knowledge}
                \textbf{Machine learning} (ML) is the process of adaptive behaviour as a result of experience (i.e. data).

                It can be divided into two types: \textbf{supervised} and \textbf{unsupervised}. Supervised stands for the user telling the program what the data means (labelled), while unsupervised for the opposite, which requires the program to identify pattern.
            \end{knowledge}

            \begin{knowledge}
                \textbf{Expert systems} trys to replicate knowledge of a human expert to diagonise a problem.

                The architecture includes
                \begin{itemize}
                    \item UI (user interface): where they ask questions and you answer;
                    \item Inference engine: deciding which question to ask next;
                    \item Rule base: linking facts; and
                    \item Knowledge base: a list of facts.
                \end{itemize}
            \end{knowledge}

    \section{Algorithm design and problem-solving}
        \begin{knowledge}
            The \textbf{program development life cycle} includes four stages,
            \begin{itemize}
                \item analysis: abstracting, decomposition, and identification of problem and requirements;
                \item design: further decomposition, creating structure diagrams, flowcharts and pseudocode;
                \item coding: writing program code and iterative testing;
                \item testing: testing program code with test data.
            \end{itemize}
        \end{knowledge}

        \begin{knowledge}
            \textbf{Sub-systems} make up computer systems and they are even made up of further sub-systems.

            We could decompose a problems into four main parts: inputs, processes, outputs and storage.

            We could use a structure diagram (which looks like a tree), a flowchart or pseudocode to design a solution.
        \end{knowledge}

        \begin{knowledge}
            \textbf{Flowchart} is a diagram representing a algorithm, and its syntax involves the following:
            \begin{itemize}
                \item Flow line, which is an arrow representing passing between states;
                \item Process, a rectangle representing some operations being done;
                \item Subroutine, a rectangle with two vertical lines close to the sides, representing calling a separate flowchart;
                \item I/O, a parallelogram representing input/output;
                \item Decision, a rhombus used to represent a Yes/No a.k.a. True/False decision being made based on a condition; and
                \item Terminator, a oval used to fill in Start or Stop.
            \end{itemize}
        \end{knowledge}

        \begin{knowledge}
            Note: time complexity not required, just for reference how quick each algorithm is.
            \begin{itemize}
                \item A \textbf{Linear Search} algorithm searches for a value (and its position) in a given list (usually unsorted, if sorted we usually use binary search which has time complexity of $\mathcal{O}(\log n)$), with time complexity of $\mathcal{O}(n)$ (not required but good to know);
                \item A \textbf{Bubble Sort} algorithm sorts elements by comparing neighbouring elements and swapping them into the correct sequence, which has a time complexity of $\mathcal{O}(n^2)$. (Faster algorithms include merge sort and quicksort which has a time complexity of $\mathcal{O}(n \log n)$.)
                \item A \textbf{totalling} algorithm finds the sum of elements in a list by iterating. It has a time complexity of $\mathcal{O}(n)$ but with pre-processing the time complexity could be reduced to $\mathcal{O}(1)$.
                \item A \textbf{counting} algorithm finds the number of elements (with a certain value) in a list. This should have a time complexity of $\mathcal{O}(n)$, but pre-processing data (during input) gives us a time complexity of $\mathcal{O}(1)$.
                \item Finding \textbf{maximum, minimum} are similar to the totalling (we do this by iterating) and has a time complexity of $\mathcal{O}(n)$ and $\mathcal{O}(1)$ without pre-processing.
                \item Finding \textbf{average} is just dividing total by number of elements.
            \end{itemize}
        \end{knowledge}

        \begin{knowledge}
            \textbf{Validation check} is the check on data to make sure is reasonable. Checks inlude:
            \begin{itemize}
                \item Range check,
                \item Length check,
                \item Type check (e.g. str or int),
                \item Presence check (e.g. NULL),
                \item Format check (e.g. 1 or one),
                \item Check digit.
            \end{itemize}
        \end{knowledge}

        \begin{knowledge}
            \textbf{Verification} is the check to make sure the input data is correct. Checks include visual check and double entry check.
        \end{knowledge}

        \begin{knowledge}
            \textbf{Testing data} is the data you input to a program to test whether it works properly or not. Types include:
            \begin{itemize}
                \item Normal: data that program should accept;
                \item Abnormal: data that program should not accept;
                \item Extreme: data at the edge of what's allowed;
                \item Boundry: data at two edges of what's allowed (just allowed and just disallowed).
            \end{itemize}
        \end{knowledge}

    \section{Programming}
        \subsection{Programming concepts}
            \begin{knowledge}
                Data types include: integer, real (a.k.a. floating point/float/double), char (character), string and boolean (T/F). They are \(\mathtt{INTEGER}, \mathtt{REAL}, \mathtt{CHAR}, \mathtt{STRING}, \mathtt{BOOLEAN}\) respectively.

                The way to declare a variable is as follows in pseudocode:
                \[
                    \mathtt{DECLARE\ <identifier>:<datatype>}
                \]
                and the way to declare a constant is:
                \[
                    \mathtt{CONSTANT\ <identifier>\ \leftarrow\ <value>}.
                \]
            \end{knowledge}
        
            \begin{knowledge}
                The sign $\leftarrow$ is called \textbf{assignment}, which is used as
                \[
                    \mathtt{<identifier>\ \leftarrow\ <value>}.
                \]
            \end{knowledge}

            \begin{knowledge}
                \textbf{Input and output} are two very important things for a program. Their syntax are
                \[
                    \mathtt{INPUT\ <identifier>}
                \]
                and
                \[
                    \mathtt{OUTPUT\ <value(s)>}.
                \]
            \end{knowledge}

            \begin{knowledge}
                \textbf{Sequence} is the idea of executing code in a certain order, it does not have a specific syntax - just don't write code the other way around.
            \end{knowledge}

            \begin{knowledge}
                \textbf{Selection} is when logical operators are use to determine which branch a program goes to. Two selection syntax are if and case, with syntax as follows:
                \begin{align*}    
                    &\mathtt{IF\ <condition>\ THEN}\\
                    &\qquad \mathtt{<statements>}\\
                    &\mathtt{ELSE\ IF\ <condition>\ THEN}\\
                    &\qquad \mathtt{<statements>}\\
                    &\ldots\\
                    &\mathtt{ELSE}\\
                    &\qquad \mathtt{<statements>}\\
                    &\mathtt{ENDIF}
                \end{align*}
                and
                \begin{align*}
                    &\mathtt{CASE\ OF\ <identifier>}\\
                    &\qquad \mathtt{<value\ 1>:<statement>}\\
                    &\qquad \mathtt{<value\ 2>:<statement>}\\
                    &\qquad \ldots\\
                    &\qquad \mathtt{OTHERWISE:<statement>}\\
                    &\mathtt{ENDCASE}
                \end{align*}

                The first is useful when a small number of selection is necessary while the second is useful when there are a lot of cases and each case only requires one statement (important! or it will look weird).

                Notice that the otherwise is optional but I strongly recommend writing it.
            \end{knowledge}

            \begin{knowledge}
                \textbf{Iteration} or \textbf{loop} is a structure when a statement (or multiple statements) are run multiple times. We have \textbf{for (count-controlled)}, \textbf{while (pre-condition)} and \textbf{repeat until (post-condition)} loops.

                \begin{align*}
                    &\mathtt{FOR\ <identifier>\ \leftarrow\ <value1>\ TO\ <value2>\ STEP\ <increment>}\\
                    &\qquad \mathtt{<statements>}\\
                    &\mathtt{NEXT\ <identifier>}
                \end{align*}

                Note that the step increment in for loop is optional and usually does not appear (unless you want to do a loop reversed).

                \begin{align*}
                    &\mathtt{WHILE\ <condition>\ DO}\\
                    &\qquad \mathtt{<statements>}\\
                    &\mathtt{ENDWHILE}
                \end{align*}

                \begin{align*}
                    &\mathtt{REPEAT}\\
                    &\qquad \mathtt{<statements>}\\
                    &\mathtt{UNTIL\ <condition>}
                \end{align*}
            \end{knowledge}

            \begin{knowledge}
                \textbf{String manipulation} includes length, substring, upper case, and lower case. We use
                \[
                    \mathtt{LENGTH(<identifier>)}
                \]
                to find a string,
                \[
                    \mathtt{LCASE(<identifier>)}
                \]
                to convert everything to lower case,
                \[
                    \mathtt{UCASE(<identifier>)}
                \]
                to convert everything to upper case, and
                \[
                    \mathtt{SUBSTRING(<identifier>, <start>, <length>)}
                \]
                to find the substring from the start position with a certain length.

                Notice that generally string starts at position 1 (by syllabus pseudocode syntax section) but the syllabus also states that you can treat it as start position 0.
            \end{knowledge}

            \begin{knowledge}
                \(\mathtt{+}, \mathtt{-}, \mathtt{*}, \mathtt{/}, \mathtt{\wedge} \text{(raised to the power of)}\) are required mathematical arithmetic operations. Furthermore,
                \[
                    \mathtt{DIV(<identifier\ 1>, <identifier\ 2>)}
                \]
                gives the integer division result (with fractional part discarded), and
                \[
                    \mathtt{MOD(<identifier\ 1>, <identifier\ 2>)}
                \]
                gives the remainder of the division result.
            \end{knowledge}

            \begin{knowledge}
                \(\mathtt{=}, \mathtt{<}, \mathtt{>}, \mathtt{<=}, \mathtt{>=}, \mathtt{<>} \text{(not equal to)}\) are required logic operators (which are used in conditions).
            \end{knowledge}

            \begin{knowledge}
                \(\mathtt{AND}, \mathtt{OR}, \mathtt{NOT}\) are required boolean operators. They can be used to combine conditions together to give a result.
            \end{knowledge}

            \begin{knowledge}
                \textbf{Nested statements} are when you put one layer of iteration/selection around another (for example, in a bubble sort).
            \end{knowledge}

            \begin{knowledge}
                \textbf{Subroutine} is a self-contained sub-module of a code (i.e. a sub-system) which can be \textbf{called} in the main program.
                
                \textbf{Procedures} operates on something while \textbf{functions} produces a return value.
                
                \textbf{Parameters} are inputs to subroutines.

                To define a procedure,
                \begin{align*}
                    &\mathtt{PROCEDURE\ <identifier> (<param1>: <datatype>, <param2>:<datatype>, \ldots)}\\
                    &\qquad \mathtt{<statements>}\\
                    &\mathtt{ENDPROCEDURE}
                \end{align*}
                and to call it,
                \[
                    \mathtt{CALL\ <identifier> (Value1, Value2, \ldots)}.
                \]

                To define a function,
                \begin{align*}
                    &\mathtt{FUNCTION\ <identifier> (<param1>: <datatype>, <param2>:<datatype>, \ldots)}\\
                    &\qquad \qquad \qquad \qquad \mathtt{RETURNS <datatype>}\\
                    &\qquad \mathtt{<statements>}\\
                    &\qquad \mathtt{RETURN <value>}\\
                    &\mathtt{ENDFUNCTION}
                \end{align*}
                and to call it,
                \[
                    \mathtt{<identifier> (Value1, Value2, \ldots)}.
                \]

            \end{knowledge}

            \begin{knowledge}
                \textbf{Local} variables are declared and used in a loop/branch/subroutine (and are invalid outside it), and \textbf{global} variables are the ones which can be used anywhere.
            \end{knowledge}

            \begin{knowledge}
                The function
                \[
                    \mathtt{ROUND(<identifier>, <places>)}
                \]
                is used to round a certain variable to a certain decimal place.

                The function
                \[
                    \mathtt{RANDOM()}
                \]
                returns a random value between 0 and 1 inclusive.
            \end{knowledge}
            
            \begin{knowledge}
                It is a good habit to make variable/constant/array/procedure/function names (identifiers) understandable, and to use procedures and functions where necessary and suitable.
            \end{knowledge}
        \subsection{Arrays}
            
            \begin{knowledge}
                \textbf{Arrays} are just simply a list of elements. \textbf{2-D Arrays} is just simply a list of lists.

                To declare a 1-D array we use
                \[
                    \mathtt{DECLARE\ <identifier>:ARRAY[<1>:<u>]\ OF\ <data\ type>}
                \]
                and use
                \[
                    \mathtt{<identifier>[<index>]}
                \]
                to call it. To declare a 2-D array we use
                \[
                    \mathtt{DECLARE\ <identifier>:ARRAY[<11>:<u1>, <12>:<u2>]\ OF\ <data\ type>}
                \]
                and use
                \[
                    \mathtt{<identifier>[<index1>, <index2>]}
                \]
                to call it.

                At a GCSE level a 2-D array is a table, but at a higher level it is not necessary.

                Notice that in the syllabus it said that beginning index can be either 0 or 1.
            \end{knowledge}

        \subsection{File Handling}
            
            \begin{knowledge}
                \textbf{File handling} is important, to save and read data easier externally into an external file. There are two types of file modes, read (\(\mathtt{READ}\)) and write (\(\mathtt{WRITE}\)).

                To open a file we write
                \[
                    \mathtt{OPENFILE <File\ identifier> FOR <File\ mode>},
                \]
                and to read/write from a file we do
                \[
                    \mathtt{READFILE <File\ identifier>, <Variable>}
                \]
                and
                \[
                    \mathtt{WRITEFILE <File\ identifier>, <Variable>}
                \]
                respectively.

                To close a file we write (it is a good habit to close after using)
                \[
                    \mathtt{CLOSEFILE <File\ identifier>}.
                \]
            \end{knowledge}

        \subsection{Sample Pseudocode for Required Algorithms}
            \begin{knowledge}
                An example code for bubble sort is:
                \begin{align*}
                    &\mathtt{FUNCTION\ BubbleSort\ (length:\ INTEGER,\ list:ARRAY [1:len]\ OF\ REAL)}\\
                    &\qquad \qquad \qquad \qquad \mathtt{RETURNS\ ARRAY [1:len]\ OF\ INTEGER}\\
                    &\qquad \mathtt{DECLARE\ swapped:BOOLEAN}\\
                    &\qquad \mathtt{swapped \leftarrow TRUE}\\
                    &\qquad \mathtt{WHILE\ swapped}\\
                    &\qquad \qquad \mathtt{swapped \leftarrow FALSE}\\
                    &\qquad \qquad \mathtt{FOR\ i\ \leftarrow\ 1\ TO\ len - 1\ DO}\\
                    &\qquad \qquad \qquad \mathtt{IF\ list[i] > list[i + 1]\ THEN}\\
                    &\qquad \qquad \qquad \qquad \mathtt{DECLARE\ tmp:REAL}\\
                    &\qquad \qquad \qquad \qquad \mathtt{tmp\ \leftarrow\ list[i + 1]}\\
                    &\qquad \qquad \qquad \qquad \mathtt{list[i + 1]\ \leftarrow\ list[i]}\\
                    &\qquad \qquad \qquad \qquad \mathtt{list[i]\ \leftarrow\ tmp}\\
                    &\qquad \qquad \qquad \qquad \mathtt{swapped \leftarrow TRUE}\\
                    &\qquad \qquad \qquad \mathtt{ENDIF}\\
                    &\qquad \qquad \mathtt{NEXT\ i}\\
                    &\qquad \mathtt{ENDWHILE}\\
                    &\qquad \mathtt{RETURN\ list}\\
                    &\mathtt{ENDFUNCTION}
                \end{align*}
            \end{knowledge}

            \begin{knowledge}
                An example code for linear seach and counting is:

                \begin{align*}
                    &\mathtt{FUNCTION\ LinearSearch\ (length:\ INTEGER,\ list:ARRAY [1:len]\ OF\ REAL,\ val:REAL)}\\
                    &\qquad \qquad \qquad \qquad \mathtt{RETURNS\ INTEGER, BOOL, INTEGER}\\
                    &\qquad \text{// This program will return the LAST index found just for counting.}\\
                    &\qquad \text{// To return the first one just RETURN after found.}\\
                    &\qquad \mathtt{DECLARE\ count:INTEGER}\\
                    &\qquad \mathtt{DECLARE\ found:BOOLEAN}\\
                    &\qquad \mathtt{DECLARE\ indexFound:INTEGER}\\
                    &\qquad \mathtt{count \leftarrow 0}\\
                    &\qquad \mathtt{found \leftarrow FALSE}\\
                    &\qquad \mathtt{indexFound \leftarrow -1}\\
                    &\qquad \mathtt{FOR\ i\ \leftarrow\ 1\ TO\ len\ DO}\\
                    &\qquad \qquad \mathtt{IF\ list[i]=val\ THEN}\\
                    &\qquad \qquad \qquad \mathtt{count \leftarrow count + 1}\\
                    &\qquad \qquad \qquad \mathtt{indexFound \leftarrow i}\\
                    &\qquad \qquad \qquad \mathtt{found \leftarrow TRUE}\\
                    &\qquad \qquad \mathtt{ENDIF}\\
                    &\qquad \mathtt{NEXT\ i}\\
                    &\qquad \mathtt{RETURN\ count, found, indexFound}\\
                    &\mathtt{ENDFUNTION}
                \end{align*}
            \end{knowledge}

            \begin{knowledge}
                An example code for finding the total, maximum, minimum, and average of a list:

                \begin{align*}
                    &\mathtt{FUNCTION\ TotMaxMinAvg\ (length:\ INTEGER,\ list:ARRAY [1:len]\ OF\ REAL)}\\
                    &\qquad \qquad \qquad \qquad \mathtt{RETURNS\ REAL, REAL, REAL, REAL}\\
                    &\qquad \mathtt{DECLARE\ tot:REAL}\\
                    &\qquad \mathtt{DECLARE\ max:REAL}\\
                    &\qquad \mathtt{DECLARE\ min:REAL}\\
                    &\qquad \mathtt{DECLARE\ avg:REAL}\\
                    &\qquad \mathtt{tot \leftarrow 0}\\
                    &\qquad \mathtt{max \leftarrow -INF}\\
                    &\qquad \mathtt{min \leftarrow INF}\\
                    &\qquad \mathtt{FOR\ i\ \leftarrow\ 1\ TO\ length\ DO}\\
                    &\qquad \qquad \mathtt{tot \leftarrow tot + list[i]}\\
                    &\qquad \qquad \mathtt{IF\ list[i] > max\ THEN}\\
                    &\qquad \qquad \qquad \mathtt{max \leftarrow list[i]}\\
                    &\qquad \qquad \mathtt{ENDIF}\\
                    &\qquad \qquad \mathtt{IF\ list[i] < min\ THEN}\\
                    &\qquad \qquad \qquad \mathtt{min \leftarrow list[i]}\\
                    &\qquad \qquad \mathtt{ENDIF}\\
                    &\qquad \mathtt{NEXT\ i}\\
                    &\qquad \mathtt{avg \leftarrow tot / len}\\
                    &\qquad \mathtt{RETURN\ tot, max, min, avg}\\
                    &\mathtt{ENDFUNCTION}
                \end{align*}
            \end{knowledge}
    \section{Databases}

        \begin{knowledge}
            A \textbf{database} is a collection of logically organised data, often arranged in \textbf{tables}.
        \end{knowledge}

        \begin{knowledge}
            The \textbf{rows} in a database is called a \textbf{record}. The \textbf{columns} in a database is called a \textbf{field} or an \textbf{attribute}.
        \end{knowledge}

        \begin{knowledge}
            A \textbf{primary key} is a field that uniquely identifies a record.
        \end{knowledge}

        \begin{knowledge}
            Similar to pseudocode, \textbf{real, integer, text (alphanumeric), character, date, time, boolean} data can be stored in databases.
        \end{knowledge}

        \begin{knowledge}
            We search for a data by \textbf{query}, using \textbf{structural query language (SQL)}.
        \end{knowledge}

        \begin{knowledge}
            A basic SQL scripts consists of the following:
            \begin{align*}
                &\mathtt{SELECT\ list\ of\ fields}\\
                &\mathtt{FROM\ name\ of\ table}\\
                &\mathtt{WHERE\ condition}\\
                &\mathtt{ORDER\ BY\ field\ ASC/DESC}
            \end{align*}
        \end{knowledge}

        \begin{knowledge}
            In the \(\mathtt{SELECT}\) field, we could also use \(\mathtt{SUM(field)}\) to find the sum of the field, and use \(\mathtt{COUNT(field)}\) to find the number of records which meet the requirements. (In fact, it finds the number of records with a non-empty field but we do not need to consider that at this stage.)
        \end{knowledge}

    \section{Boolean logic}
        \begin{knowledge}
            We are expected to know the following logic symbols:
            \begin{itemize}
                \item NOT. A verticle line at the input, a tip with a circle at the output.
                \item AND. A vertical line at the input and a curve at the output.
                \item OR. A curve at the input and a curve at the output.
                \item NAND. AND gate with a circle on the output.
                \item NOR. OR gate with a circle on the output.
                \item XOR (EOR). (E stands for Exclusive) OR gate with an extra curve at the input.
            \end{itemize}
        \end{knowledge}

        \begin{knowledge}
            Only NOT gate has a single input; all the rest has two inputs.
        \end{knowledge}

        \begin{knowledge}
            The logic gates create the following outputs:
            \begin{itemize}
                \item NOT. 1 iff 0.
                \item AND. 1 iff (1, 1).
                \item OR. 0 iff (0, 0).
                \item NAND. 0 iff (1, 1).
                \item NOR. 1 iff (0, 0).
                \item XOR. 1 iff different.
            \end{itemize}
            Note: iff stands for if and only if.
        \end{knowledge}

    \section*{Afterwords}
        I hate typing Pseudocode like this in \LaTeX!

        This sheet really took me some time to populate, but I would genuinely like to share my understanding of GCSE CS (and beyond) with all of you.
        
        My deeps thanks to all Computer Science teachers I have met and my friends who helped and supported me with producing this.
        
        I would like to give special thanks to Dr. Mulvey, my current Computer Science teacher, and Dr. Zhang, my tutor for providing me with all this knowledge.

        Finally, I really hope this helped you to gain a better understanding of Physics. Feel free to email \href{eason.syc@icloud.com}{eason.syc@icloud.com} to send me a feedback.

\end{document}